\documentclass{article}
\usepackage[utf8]{inputenc}
\usepackage{amsmath,amssymb}
\usepackage{enumitem}
\usepackage{graphicx,tikz}
\usepackage{geometry}

\geometry{margin=1in}
\pagestyle{empty}
\setlength{\parindent}{0pt}
\setlength{\parskip}{0.5em}

\newcommand{\code}[1]{{\fontfamily{pcr}\selectfont #1}}
\newcommand{\abs}[1]{\left|#1\right|}
\newcommand{\srs}[1]{\{#1\}_1^\infty}
\newcommand{\floor}[1]{\left\lfloor#1\right\rfloor}
\newcommand{\ceil}[1]{\left\lceil#1\right\rceil}

\newcommand{\env}[2]{\begin{#1}#2\end{#1}}
\newcommand{\spl}[1]{\begin{split}#1\end{split}}
\newcommand{\mat}[1]{\begin{bmatrix}#1\end{bmatrix}}

\newcommand{\zee}{\mathbb{Z}}
\newcommand{\Q}{\mathbb{Q}}
\newcommand{\arr}{\mathbb{R}}
\newcommand{\C}{\mathbb{C}}
\newcommand{\N}{\mathbb{N}}

\begin{document}

\setcounter{section}{1}
\section{Introduction to Modular Arithmetic}

This section introduces the ideas and notation we will use regarding
modular arithmetic.
Theorems are presented without proof,
as they are all already well-established.

\textbf{Definition 2.1.} For a positive integer $n$, the set of integers
modulo $n$ is denoted $\zee_n$.

\textbf{Theorem 2.1.} For any positive integer $n$, $\zee_n$ is a ring.

\textbf{Theorem 2.2.} For any prime $p$, $\zee_p$ is a field.

\textbf{Definition 2.2.} For any $k, n \in \N$, the \textit{order
of $k$ modulo $n$} is the least $r \in \N$ such that
$k^r \equiv 0$ or 1 $\mod n$.

\textbf{Definition 2.3.} The \textit{Euler totient} of a
positive integer $n$,
denoted $\phi(n)$, is the number of positive integers less than $n$
that are coprime to $n$.

\textbf{Theorem 2.3.} For any prime $p$, $\phi(p) = p - 1$.

\textbf{Theorem 2.4.} For any $k \in \zee, n \in \N$, if $r$ is the order of
$k$ modulo $n$, then $r \leq \phi(n)$.

\textbf{Theorem 2.5.} For any $k \in \zee, n \in \N$ with
$\gcd(k, n) = 1$, there exists no $m \in \zee$ such that
$km \equiv 0 \mod n$.

\textbf{Definition 2.4.} We call a positive integer $k$ a
\textit{primitive root} modulo $n$ if the order of $k$ modulo $n$
is $\phi(n)$.

\section{$\oplus_k$ over $\zee_p$}

\textbf{Definition 3.1.} $\zee_p'$ denotes the set of all
$A \in \zee_p$ for which $0 \not\in A$.

\textbf{Definition 3.2.} Let $A \in \zee_p'$. For all $i \in \zee$
with $0 \leq i \leq p - 2$, let $z_i = 0$ if $k^i \not\in A$,
and let $z_i = 1$ if $k^i \in A$. Then, define
$\beta: \mathcal{P}(\zee_p') \rightarrow \zee$ by
\[\beta(A) = \sum_{i = 0}^{p-2} z_i \cdot 2^{i}\]
In other words, consider the binary representation of $\beta(A)$.
The least significant bit is 1 if $k^0 \in A$, and 0 otherwise.
The next bit is 1 if $k^1 \in A$ and 0 otherwise.
This continues to the most significant bit,
which is 1 if $k^{p-2} \in A$ and 0 otherwise.

Consider the bitwise operators $\&$ (AND) and $\wedge$ (XOR).
Let $L: \zee \rightarrow \zee$ denote left rotation by one bit
over the $p-1$ least significant bits.
Under this construction,
\[\beta(A \cup B - A \cap B) = \beta(A) \wedge \beta(B)\]
and
\[\beta(\{mk \mid m \in A \cap B\}) = L(\beta(A) \mathbin{\&} \beta(B))\]

\textbf{Definition 3.3.} Consider the bitwise operators $\&$ (AND)
and $\wedge$ (XOR).
Let $L: \zee \rightarrow \zee$ denote left rotation by one bit
over the $p-1$ least significant bits.
Define $\gamma: \zee \rightarrow \zee$
by
\[\gamma(x, y) = (x \wedge y, L(x \mathbin{\&} y))\]

Note that repeated applications of $\gamma$ are equivalent to
repeated applications of $\eta_k$, so for some $N \in \N$,
\[\pi_1 \circ \gamma^N(\beta(A), \beta(B)) = \beta(A \oplus_k B)\]

\textbf{Definition 3.4.} Define $\zeta: \mathcal{P}(\zee_p')
\rightarrow \zee[x]/(x^{2^{p-1}}-x)$ by
\[\zeta(A) = x^{\beta(A)}\]

\newpage
\textbf{Theorem 3.1.} \textit{Let $A, B \subseteq \zee_p'$.
Then, $\zeta(A) \cdot \zeta(B) = \zeta(A \oplus_k B)$.}

Let $x, y \in \zee$ with $0 \leq x < 2^{p-1}$
and $0 \leq y < 2^{p-1}$.
Consider the following recursive definition of addition:
\[x + y = (x \wedge y) + 2(x \mathbin{\&} y)\]
Note that if $x + y < 2^{p-1}$, then
$2(x \mathbin{\&} y) < 2^{p-1}$, so the $(p-1)$th
least siginificant bit of $x \mathbin{\&} y$ is zero,
and $2(x \mathbin{\&} y) = L(x \mathbin{\&} y)$.
Therefore, in the case that $\beta(A) + \beta(B) < 2^{p-1}$,
we find
$\beta(A) + \beta(B) = \beta(A \oplus_k B)$,
which proves the theorem.

If $x + y \geq 2^{p-1}$, then after some number of iterations,
we will find $x', y' \in \zee$ for which
$2^{p-1} \leq 2(x \mathbin{\&} y) < 2^p$.
Note that, since $x + y < 2 \cdot 2^{p-1} = 2^p$, this will occur exactly
once in all the iterations. In this case,
$2(x \mathbin{\&} y) - 2^{p-1} + 1 = L(x \mathbin{\&} y)$.
Therefore, if $\beta(A) + \beta(B) \geq 2^{p-1}$, then
$\beta(A \oplus_k B) = \beta(A) + \beta(B) - 2^{p-1} + 1$.
Thus,
\[\begin{split}
    \zeta(A \oplus_k B) &= x^{\beta(A) + \beta(B) - 2^{p-1} + 1} \\
    &= x^{\beta(A) + \beta(B) - 2^{p-1} + 1} +
        x^{\beta(A) + \beta(B) - 2^{p-1}}(x^{2^{p-1}} - x) \\
    &= x^{\beta(A) + \beta(B) - 2^{p-1} + 1} +
        x^{\beta(A) + \beta(B)} - x^{\beta(A) + \beta(B) - 2^{p-1} + 1} \\
    &= x^{\beta(A) + \beta(B)} \\
    &= \zeta(A) \cdot \zeta(B)
\end{split}\]
This completes the proof of the theorem. $\blacksquare$

\end{document}
