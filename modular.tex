\documentclass{article}
\usepackage[utf8]{inputenc}
\usepackage{amsmath,amssymb}
\usepackage{enumitem}
\usepackage{graphicx,tikz}
\usepackage{geometry}

\geometry{margin=1in}
\pagestyle{empty}
\setlength{\parindent}{0pt}
\setlength{\parskip}{0.5em}

\newcommand{\code}[1]{{\fontfamily{pcr}\selectfont #1}}
\newcommand{\abs}[1]{\left|#1\right|}
\newcommand{\srs}[1]{\{#1\}_1^\infty}
\newcommand{\floor}[1]{\left\lfloor#1\right\rfloor}
\newcommand{\ceil}[1]{\left\lceil#1\right\rceil}

\newcommand{\env}[2]{\begin{#1}#2\end{#1}}
\newcommand{\spl}[1]{\begin{split}#1\end{split}}
\newcommand{\mat}[1]{\begin{bmatrix}#1\end{bmatrix}}

\newcommand{\zee}{\mathbb{Z}}
\newcommand{\Q}{\mathbb{Q}}
\newcommand{\arr}{\mathbb{R}}
\newcommand{\C}{\mathbb{C}}
\newcommand{\N}{\mathbb{N}}

\begin{document}

\setcounter{section}{2}
\section{Introduction to Modular Arithmetic}

This section introduces the ideas and notation we will use regarding
modular arithmetic.
Theorems are presented without proof,
as they are all already well-established.

\textbf{Definition 3.1.} For a positive integer $n$, the set of integers
modulo $n$ is denoted $\zee_n$.

\textbf{Theorem 3.1.} For any positive integer $n$, $\zee_n$ is a ring.

\textbf{Theorem 3.2.} For any prime $p$, $\zee_p$ is a field.

\textbf{Definition 3.2.} For any $k, n \in \N$, the \textit{order
of $k$ modulo $n$} is the least $r \in \N$ such that
$k^r \equiv 0$ or 1 $\mod n$.

\textbf{Definition 3.3.} The \textit{Euler totient} of a
positive integer $n$,
denoted $\phi(n)$, is the number of positive integers less than $n$
that are coprime to $n$.

\textbf{Theorem 3.3.} For any prime $p$, $\phi(p) = p - 1$.

\textbf{Theorem 3.4.} For any $k \in \zee, n \in \N$, if $r$ is the order of
$k$ modulo $n$, then $r \leq \phi(n)$.

\textbf{Theorem 3.5.} For any $k \in \zee, n \in \N$ with
$\gcd(k, n) = 1$, there exists no $m \in \zee$ such that
$km \equiv 0 \mod n$.

\textbf{Definition 3.4.} We call a positive integer $k$ a
\textit{primitive root} modulo $n$ if the order of $k$ modulo $n$
is $\phi(n)$.

\section{$\oplus_k$ over $\zee_n$}

Let $n \in \N$, and let $k \in \zee_n$ with $\gcd(k, n) = 1$.

\textbf{Definition 4.1.} For any $S \subseteq \zee_n$ and $r \in \N$,
$S^r = \underbrace{S \oplus_k S \oplus_k \ldots \oplus_k S}
_{r\text{ times}}$

\textbf{Definition 4.2.} For any $S \subseteq \zee_n$, the \textit{order}
of $S$ is the least $r \in \N$ such that $\exists r' \in \N$ with
$r' < r$ and $S^{r'} = S^r$.

\textbf{Lemma 4.1.} \textit{For any $S \in \zee_n$,
$S^2 = \{sk \mid s \in S\}$.}

Let $S \subseteq \zee_n$.
Then, $S \cap S = S$, and $S \cup S - S \cap S = \emptyset$.
Thus,
\[\begin{split}
S^2 &= S \oplus_k S \\
&= (S \cup S - S \cap S) \oplus_k \{sk \mid s \in S \cap S\} \\
&= \emptyset \oplus_k \{sk \mid s \in S\} \\
&= \{sk \mid s \in S\}
\end{split}\]
This proves the lemma. $\blacksquare$

\textbf{Theorem 4.1.} \textit{Let $r$ be the order of $k$ modulo $n$.
Then, the order of a set $S \subseteq \zee_n$ cannot exceed $2^r$.}

Suppose associativity holds. Let $S \subseteq \zee_n$.
Then,
\[\begin{split}
    S^{2^r} &= (S^2)^{2^{r-1}} \\
    \text{(by Lemma 4.1)} &= (\{sk \mid s \in S\})^{2^{r-1}} \\
    &= ((\{sk \mid s \in S\})^2)^{2^{r-2}} \\
    \text{(by Lemma 4.1)} &= (\{sk^2 \mid s \in S\})^{2^{r-2}}
\end{split}\]
By repeated applications of Lemma 4.1, we find
\[\begin{split}
    S^{2^r} &= \{sk^r \mid s \in S\}^{2^{r-r}} \\
    &= \{sk^r \mid s \in S\}
\end{split}\]
Since $r$ is the order of $k$ modulo $n$, $k^r \equiv 0$ or $1 \mod n$.
By Theorem 3.5, $k^r = k \cdot k^{r-1} \not\equiv 0 \mod n$, so
$k^r \equiv 1 \mod n$. Therefore,
\[\begin{split}
    S^{2^r} &= \{sk^r \mid s \in S\} \\
    &= \{s \mid s \in S\} \\
    &= S
\end{split}\]
This proves the theorem. $\blacksquare$
\footnote{This depends on associativity, which has not yet been proven.}

\textbf{Definition 4.3.} Let $A \subseteq \zee_n$, and let $b \in \zee_n$.
The \textit{prevalence of $b$ in $A$} is the least $v \in \N$ such
that $bk^v \not\in A$. If no such $v$ exists, it is the order
of $k$ modulo $n$.

\textbf{Definition 4.4.} Let $A \subseteq \zee_n$, and let $b \in \zee_n$
with prevalence $v$ in $A$.
The \textit{prevalence set of $b$ in $A$},
denoted $P(b, A)$, is $\{b, bk, \ldots, bk^v\}$.

\textbf{Lemma 4.2.} \textit{Let $A \subseteq \zee_n$, and
let $b \in \zee_n$ with prevalence $v \in \N$ in $A$.
Then,
$A \oplus_k \{b\} = (A - P(b, A)) \cup \{bk^v\}$.}

We will perform induction on $v$. Suppose $v = 1$.
Then, $b^1 = b \not\in A$, so $A \cap \{b\} = \emptyset$.
Therefore, by Lemma 2.1,
\[A \oplus_k \{b\} = A \cup \{b\} = (A - \{b\}) \cup \{b\}
= (A - P(b, A)) \cup \{b\}\]
This proves the base case.

Suppose the lemma holds when the prevalence is $v-1$. Then,
\[\begin{split}
    A \oplus_k \{b\} &=
    (A \cup \{b\} - A \cap \{b\})
        \oplus_k \{mk \mid m \in A \cap \{b\}\} \\
    &= (A - \{b\}) \oplus_k \{bk\}
\end{split}\]
If $v$ is the order of $k$ modulo $n$, then
$bk \cdot k^{v-1} = bk^v = b \not\in (A - \{b\})$,
so $v-1$ is the prevalence of $bk$ in $A - \{b\}$.
Otherwise, $bk \cdot k^{v-1} = bk^v \not\in (A - \{b\})$,
so $v-1$ is still the prevalence of $bk$ in
$A - \{b\}$. Therefore, by the inductive hypothesis,
\[\begin{split}
    A \oplus_k \{b\}
    &= (A - \{b\}) \oplus_k \{bk\} \\
    &= ((A - \{b\})
        - \{bk, bk \cdot k, \ldots, bk \cdot k^{v-1}\})
        \cup \{bk \cdot k^{v-1}\} \\
    &= (A - \{b, bk, \ldots, bk^v\}) \cup \{bk^v\} \\
    &= (A - P(b, A)) \cup \{bk^v\}
\end{split}\]
This completes the inductive step, proving the lemma. $\blacksquare$

\textbf{Lemma 4.3.} \textit{Let $A \subseteq \zee_n$, and
let $b, c \in \zee_n$.
If $P(b, A) \cap P(c, A) \neq \emptyset$, then
either $P(b, A) \subseteq P(c, A)$ or $P(c, A) \subseteq P(b, A)$.}

Let $u, v \in \N$ be the prevalences of $b$ and $c$ respectively in $A$.
We can suppose without loss of generality that $u \geq v$---if it is not,
we swap $b$ and $c$.
Suppose $P(b, A) \cap P(c, A) \neq \emptyset$. Then,
$\exists i, j \in \N$ with $i \leq u$ and $j \leq v$
such that $bk^i = ck^j$. Since $bk^i \in P(b, A)$, the prevalence of
$bk^i$ in $A$ is $u - i$. Similarly, the prevalence of $ck^j$ in $A$
is $v - j$, so we have $u - i = v - j$. Since $u \geq v$, we find
$i \geq j$.

After simplifying $bk^i = ck^j$, we have $bk^{i-j} = c$. Therefore,
$P(c, A) = \{bk^{i-j}, bk^{i-j+1}, \ldots bk^u\} \subseteq P(b, A)$.
Removing the condition that $u \leq v$, we have
$P(b, A) \subseteq P(c, A)$ or $P(c, A) \subseteq P(b, A)$,
proving the lemma.
$\blacksquare$

\newpage
\textbf{Lemma 4.4.} \textit{Let $A \in \zee_n$, and let $b, c \in R$.
Then,
$(A \oplus_k \{b\}) \oplus_k \{c\}
= (A \oplus_k \{c\}) \oplus_k \{b\}$.}

If $P(b, A) \cap P(c, A) = \emptyset$, then
$P(c, A) = P(c, (A - P(b, A)) \cup \{bk^u\})$
and $P(b, A) = P(b, (A - P(c, A)) \cup \{ck^v\})$. Therefore,
\[\begin{split}
    (A \oplus_k \{b\}) \oplus_k \{c\}
    &= ((A - P(b, A)) \cup \{bk^u\}) \oplus_k c \\
    &= (((A - P(b, A)) \cup \{bk^u\})
        - P(c, (A - P(b, A)) \cup \{bk^u\})) \cup \{ck^v\} \\
    &= (((A - P(b, A)) \cup \{bk^u\}) - P(c, A)) \cup \{ck^v\} \\
    &= (A - P(b, A) - P(c, A)) \cup \{bk^u, ck^v\} \\
    &= (((A - P(c, A)) \cup \{ck^v\}) - P(b, A)) \cup \{bk^u\} \\
    &= (((A - P(c, A)) \cup \{ck^v\})
        - P(b, (A - P(c, A)) \cup \{ck^v\})) \cup \{bk^u\} \\
    &= ((A - P(c, A)) \cup \{ck^v\}) \oplus_k \{b\} \\
    &= (A \oplus_k \{c\}) \oplus_k \{b\}
\end{split}\]
This proves the lemma when $P(b, A) \cap P(c, A) = \emptyset$.

Suppose $P(b, A) \cap P(c, A) \neq \emptyset$.
Let $u, v \in \N$ be the prevalences of $b$ and $c$ respectively in $A$.
We can again suppose without loss of generality that $u \geq v$.
By Lemma 4.2, we have
\[(A \oplus_k \{b\}) \oplus_k \{c\}
= ((A - P(b, A)) \cup \{bk^u\}) \oplus_k \{c\}\]
Since $P(c, A) \subseteq P(b, A)$, $c \in P(b, A)$.
Maybe this is going somewhere? $\blacksquare$

\textbf{Lemma 4.5.} \textit{Let $A, B \subseteq \zee_n$,
and let $c \in \zee_n$. Then,
$(A \oplus_k \{c\}) \oplus_k B = A \oplus_k (B \oplus_k \{c\})$).}


\end{document}
