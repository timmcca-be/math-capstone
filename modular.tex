\documentclass{article}
\usepackage[utf8]{inputenc}
\usepackage{amsmath,amssymb}
\usepackage{enumitem}
\usepackage{graphicx,tikz}
\usepackage{geometry}

\geometry{margin=1in}
\pagestyle{empty}
\setlength{\parindent}{0pt}
\setlength{\parskip}{0.5em}

\newcommand{\code}[1]{{\fontfamily{pcr}\selectfont #1}}
\newcommand{\abs}[1]{\left|#1\right|}
\newcommand{\srs}[1]{\{#1\}_1^\infty}
\newcommand{\floor}[1]{\left\lfloor#1\right\rfloor}
\newcommand{\ceil}[1]{\left\lceil#1\right\rceil}

\newcommand{\env}[2]{\begin{#1}#2\end{#1}}
\newcommand{\spl}[1]{\begin{split}#1\end{split}}
\newcommand{\mat}[1]{\begin{bmatrix}#1\end{bmatrix}}

\newcommand{\zee}{\mathbb{Z}}
\newcommand{\Q}{\mathbb{Q}}
\newcommand{\arr}{\mathbb{R}}
\newcommand{\C}{\mathbb{C}}
\newcommand{\N}{\mathbb{N}}

\begin{document}

\setcounter{section}{2}
\section{Introduction to Modular Arithmetic}

This section introduces the ideas and notation we will use regarding
modular arithmetic.
Theorems are presented without proof,
as they are all already well-established.

\textbf{Definition 3.1.} For a positive integer $n$, the set of integers
modulo $n$ is denoted $\zee_n$.

\textbf{Theorem 3.1.} For any positive integer $n$, $\zee_n$ is a ring.

\textbf{Theorem 3.2.} For any prime $p$, $\zee_p$ is a field.

\textbf{Definition 3.2.} For any $k, n \in \N$, the \textit{order
of $k$ modulo $n$} is the least $r \in \N$ such that
$k^r \equiv 0$ or 1 $\mod n$.

\textbf{Definition 3.3.} The \textit{Euler totient} of a
positive integer $n$,
denoted $\phi(n)$, is the number of positive integers less than $n$
that are coprime to $n$.

\textbf{Theorem 3.3.} For any prime $p$, $\phi(p) = p - 1$.

\textbf{Theorem 3.4.} For any $k \in \zee, n \in \N$, if $r$ is the order of
$k$ modulo $n$, then $r \leq \phi(n)$.

\textbf{Theorem 3.5.} For any $k \in \zee, n \in \N$ with
$\gcd(k, n) = 1$, there exists no $m \in \zee$ such that
$km \equiv 0 \mod n$.

\textbf{Definition 3.4.} We call a positive integer $k$ a
\textit{primitive root} modulo $n$ if the order of $k$ modulo $n$
is $\phi(n)$.

\section{$\oplus_k$ over $\zee_n$}

Let $n \in \N$, and let $k \in \zee_n$.

\textbf{Definition 4.1.} For any $S \subseteq \zee_n$ and $r \in \N$,
$S^r = \underbrace{S \oplus_k S \oplus_k \ldots \oplus_k S}
_{r\text{ times}}$

\textbf{Definition 4.2.} For any $S \subseteq \zee_n$, the \textit{order}
of $S$ is the least $r \in \N$ such that $\exists r' \in \N$ with
$r' < r$ and $S^{r'} = S^r$.

\textbf{Lemma 4.1.} \textit{For any $S \in \zee_n$,
$S^2 = \{sk \mid s \in S\}$.}

Let $S \subseteq \zee_n$.
Then, $S \cap S = S$, and $S \cup S - S \cap S = \emptyset$.
Thus,
\[\begin{split}
S^2 &= S \oplus_k S \\
&= (S \cup S - S \cap S) \oplus_k \{sk \mid s \in S \cap S\} \\
&= \emptyset \oplus_k \{sk \mid s \in S\} \\
&= \{sk \mid s \in S\}
\end{split}\]
This proves the lemma. $\blacksquare$

\textbf{Theorem 4.1.} \textit{Suppose $\gcd(k, n) = 1$.
Let $r$ be the order of $k$ modulo $n$.
Then, the order of a set $S \subseteq \zee_n$ cannot exceed $2^r$.}

Suppose associativity holds. Let $S \subseteq \zee_n$.
Then,
\[\begin{split}
    S^{2^r} &= (S^2)^{2^{r-1}} \\
    \text{(by Lemma 4.1)} &= (\{sk \mid s \in S\})^{2^{r-1}} \\
    &= ((\{sk \mid s \in S\})^2)^{2^{r-2}} \\
    \text{(by Lemma 4.1)} &= (\{sk^2 \mid s \in S\})^{2^{r-2}}
\end{split}\]
By repeated applications of Lemma 4.1, we find
\[\begin{split}
    S^{2^r} &= \{sk^r \mid s \in S\}^{2^{r-r}} \\
    &= \{sk^r \mid s \in S\}
\end{split}\]
Since $r$ is the order of $k$ modulo $n$, $k^r \equiv 0$ or $1 \mod n$.
By Theorem 3.5, $k^r = k \cdot k^{r-1} \not\equiv 0 \mod n$, so
$k^r \equiv 1 \mod n$. Therefore,
\[\begin{split}
    S^{2^r} &= \{sk^r \mid s \in S\} \\
    &= \{s \mid s \in S\} \\
    &= S
\end{split}\]
This proves the theorem. $\blacksquare$
\footnote{This depends on associativity, which has not yet been proven.}

\end{document}
