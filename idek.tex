\documentclass{article}
\usepackage[utf8]{inputenc}
\usepackage{amsmath,amssymb}
\usepackage{enumitem}
\usepackage{graphicx,tikz}
\usepackage{geometry}

\geometry{margin=1in}
\pagestyle{empty}
\setlength{\parindent}{0pt}
\setlength{\parskip}{0.5em}

\newcommand{\code}[1]{{\fontfamily{pcr}\selectfont #1}}
\newcommand{\abs}[1]{\left|#1\right|}
\newcommand{\srs}[1]{\{#1\}_1^\infty}
\newcommand{\floor}[1]{\left\lfloor#1\right\rfloor}
\newcommand{\ceil}[1]{\left\lceil#1\right\rceil}

\newcommand{\env}[2]{\begin{#1}#2\end{#1}}
\newcommand{\spl}[1]{\begin{split}#1\end{split}}
\newcommand{\mat}[1]{\begin{bmatrix}#1\end{bmatrix}}

\newcommand{\zee}{\mathbb{Z}}
\newcommand{\Q}{\mathbb{Q}}
\newcommand{\arr}{\mathbb{R}}
\newcommand{\C}{\mathbb{C}}
\newcommand{\N}{\mathbb{N}}

\begin{document}

\textbf{Lemma.} \textit{Let $a \in \zee_n$.
Then, $\exists d \in \zee$
such that $\{1\}^d = \{a\}$.}

let $r$ be the least
non-negative integer for which $k^r \equiv a \mod n$.
We will perform induction on $r$.

Suppose $r = 0$. Then,
$a = 1$, so $\{a\}^1 = \{1\}$. This proves the base case.

Suppose the hypothesis holds for $r - 1$. Then,
$\{1\}^{2^{r-1}} = \{k^{r-1}\}$,
so
\[\{1\}^{2^r} = \{k^{r-1}\} \oplus_k \underbrace{
    \{1\} \oplus_k \{1\} \oplus_k \ldots \oplus_k \{1\}}
_{2^{r-1}\text{ times}}\]

\end{document}
