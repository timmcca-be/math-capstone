\documentclass{article}
\usepackage[utf8]{inputenc}
\usepackage{amsmath,amssymb}
\usepackage{enumitem}
\usepackage{graphicx,tikz}
\usepackage{geometry}

\geometry{margin=1in}
\pagestyle{empty}
\setlength{\parindent}{0pt}
\setlength{\parskip}{0.5em}

\newcommand{\code}[1]{{\fontfamily{pcr}\selectfont #1}}
\newcommand{\abs}[1]{\left|#1\right|}
\newcommand{\srs}[1]{\{#1\}_1^\infty}
\newcommand{\floor}[1]{\left\lfloor#1\right\rfloor}
\newcommand{\ceil}[1]{\left\lceil#1\right\rceil}
\newcommand{\sumt}{\text{sum}}

\newcommand{\env}[2]{\begin{#1}#2\end{#1}}
\newcommand{\spl}[1]{\begin{split}#1\end{split}}
\newcommand{\mat}[1]{\begin{bmatrix}#1\end{bmatrix}}

\newcommand{\zee}{\mathbb{Z}}
\newcommand{\Q}{\mathbb{Q}}
\newcommand{\arr}{\mathbb{R}}
\newcommand{\C}{\mathbb{C}}
\newcommand{\N}{\mathbb{N}}

\begin{document}

Let $R$ be a ring, and let $k \in R$.

\section{Introduction of $\eta_k$ and $\oplus_k$}

\textbf{Definition 1.1.} For $A, B \in \mathcal{P}(R)$, we define
$\eta_k: \mathcal{P}(R) \times \mathcal{P}(R)
\rightarrow \mathcal{P}(R) \times \mathcal{P}(R)$ as
\[\eta_k(A, B) = ((A \cup B) - (A \cap B), \{mk \mid m \in (A \cap B)\})\]
For $0 \leq n \in \N$, $\eta_k^{(n)}$ denotes $\eta_k$ composed
with itself $n$ times. $\pi_1 \circ \eta_k^{(n)}$ denotes the first
element of the output of $\eta_k^{(n)}$, and $\pi_2 \circ \eta_k^{(n)}$
denotes the second.

\textbf{Lemma 1.1.} \textit{
    For any $A \in \mathcal{P}(R)$ and for any $n \in \zee$
    with $n \geq 0$,
    $\eta_k^{(n)}(A, \emptyset) = (A, \emptyset)$.
}

Suppose $n = 0$. Trivially,
$\eta_k^{(0)}(A, \emptyset) = (A, \emptyset)$.

For $n > 0$, we will use induction on $n$.
Suppose $n = 1$. Then,
\[\begin{split}
    \eta_k(A, \emptyset)
    &= ((A \cup \emptyset) - (A \cap \emptyset),
    \{mk \mid m \in (A \cap \emptyset)\}) \\
    &= (A - \emptyset, \{mk \mid m \in \emptyset\}) \\
    &= (A, \emptyset)
\end{split}\]
This proves the base case.

Now, suppose the hypothesis holds
for $n$. We will show that it holds for $n+1$.
\[\begin{split}
    \eta_k^{(n+1)}(A, \emptyset)
    &= \eta_k(\eta_k^{(n)}(A, \emptyset)) \\
    \text{(by inductive hypothesis)}\quad
    &= \eta_k(A, \emptyset) \\
    \text{(by $n=1$ case)}\quad
    &= (A, \emptyset)
\end{split}\]
This proves the inductive step, completing the proof of the
lemma. $\blacksquare$

\textbf{Lemma 1.2.} \textit{
    For any $A, B \in \mathcal{P}(R)$, if there exists a
    nonnegative integer $N$ such that
    $\pi_2 \circ \eta_k^{(N)} = \emptyset$,
    then $\exists C \in \mathcal{P}(R)$ such that,
    for any integer $n$ with $n \geq N$,
    $\eta_k^{(n)} = (C, \emptyset)$.
}

Let $C = \pi_1 \circ \eta_k^{(N)}(A, B)$. Then,
\[\begin{split}
    \eta_k^{(n)}(A, B)
    &= \eta_k^{(n-N)} \circ \eta_k^{(N)}(A, B) \\
    &= \eta_k^{(n-N)}\big(
        \pi_1 \circ \eta_k^{(N)}(A, B), \emptyset
    \big) \\
    &= \eta_k^{(n-N)}(C, \emptyset) \\
    \text{(by Lemma 1.1)}\quad
    &= (C, \emptyset)
\end{split}\]
This proves the lemma. $\blacksquare$

\textbf{Definition 1.2.} $\mathcal{P}_f(R)$ denotes the
set of all elements of $\mathcal{P}(R)$ which contain
a finite number of elements.

\textbf{Theorem 1.1 (Convergence of $\eta_k^{(n)}$).} \textit{
    For any $A, B \in \mathcal{P}_f(R)$, there exists
    some nonnegative integer $N$
    such that, for some $C \in \mathcal{P}_f(R)$,
    $N \leq n \in \zee \implies \eta_k^{(n)}(A, B) = (C, \emptyset)$.
}

We will use contradiction to show that there exists a nonnegative
integer $N$ for which $\pi_2 \circ \eta_k^{(N)} = \emptyset$.
By Lemma 1.2, this statement is equivalent to Theorem 1.1.

Suppose $A, B \in \mathcal{P}(R)$ for which no such $N$ exists.
Let $A', B' \in \mathcal{P}(R)$ with $\eta_k(A, B) = (A', B')$.
If $A \cap B = \emptyset$, then $\pi_2 \circ \eta_k(A, B)
= B' = \emptyset$.
This is a contradiction, so we will assume $A \cap B$ is not empty.

We will now consider the size of $A'$
\[\begin{split}
    \abs{A'} &= \abs{A \cup B} - \abs{A \cap B}
    = (\abs{A} + \abs{B} - \abs{A \cap B}) - \abs{A \cap B} \\
    &= \abs{A} + \abs{B} - 2\abs{A \cap B}
\end{split}\]
And of $B'$
\[\abs{B'} = \abs{A \cap B}\]
Adding these together, we find
\[\begin{split}
\abs{A'} + \abs{B'}
&= \abs{A} + \abs{B} - 2\abs{A \cap B} + \abs{A \cap B}
= \abs{A} + \abs{B} - \abs{A \cap B} \\
&< \abs{A} + \abs{B}
\end{split}\]
If there exists a nonnegative integer $N$ for which
$\pi_2 \circ \eta_k^{(N)}(A', B') = \emptyset$, then
$\pi_2 \circ \eta_k^{(N+1)}(A, B) = \emptyset$. This is
a contradiction, so no such $N$ exists for $A'$ and $B'$.
Therefore, we can repeatedly apply the same reasoning, replacing
$A$ and $B$ with $A'$ and $B'$. Doing so, we find
\[\abs{A} + \abs{B}
> \abs{\pi_1 \circ \eta_k(A, B)} + \abs{\pi_1 \circ \eta_k(A, B)}
> \abs{\pi_1 \circ \eta_k^{(2)}(A, B)}
    + \abs{\pi_1 \circ \eta_k^{(2)}(A, B)}
> \ldots\]
For all $n \in \zee$ with $n \geq 0$, as $n$ increases,
$f(n) = \abs{\pi_1 \circ \eta_k^{(n)}(A, B)}
+ \abs{\pi_1 \circ \eta_k^{(n)}(A, B)}$ strictly decreases.
However, it is an integer and is bounded above by
$\abs{A} + \abs{B}$, so for some value of $n$, $f(n) < 0$.
Since the size of a set can never be negative, this is a contradiction.
Therefore, there must exist some nonnegative integer $N$ for which
$\pi_2 \circ \eta_k^{(N)} = \emptyset$. This proves the theorem.
$\blacksquare$

\textbf{Corollary 1.1.} \textit{
    For any $A, B \in \mathcal{P}_f(R)$, let $N$ be the
    least nonnegative integer such that
    $\pi_2 \circ \eta_k^{(N)} = \emptyset$. Then,
    $\exists C \in \mathcal{P}_f(R)$ such that
    $N \leq n \in \zee \implies \eta_k^{(n)}(A, B) = (C, \emptyset)$.
}

By Theorem 1.1, such a value exists of $N$ exists.
Since $N$ is bounded below by 0, we can find a minimum value for which
$\pi_2 \circ \eta_k^{(N)} = \emptyset$. Applying Lemma 1.2 to the
minimum value proves the corollary. $\blacksquare$

\textbf{Definition 1.3.} For $A, B \in \mathcal{P}_f(R)$,
let $N$ be the least value for which
$\pi_2 \circ \eta_k^{(N)}(A, B) = \emptyset$. Then, we define
the binary operator $\oplus_k: \mathcal{P}_f(R) \times \mathcal{P}_f(R)
\rightarrow \mathcal{P}_f(R)$ as
\[A \oplus_k B = \pi_1 \circ \eta_k^{(N)}(A, B)\]
This is equivalent to the following recursive definition:
\[A \oplus_k B = \begin{cases}
    A & B = \emptyset \\
    ((A \cup B) - (A \cap B)) \oplus_k \{mk \mid m \in (A \cap B)\}
        & B \neq \emptyset
\end{cases}\]
Also note that by Lemma 1.1, in any case,
\[A \oplus_k B
= ((A \cup B) - (A \cap B)) \oplus_k \{mk \mid m \in (A \cap B)\}\]

\section{Properties of $\oplus_k$}

\textbf{Theorem 2.1 (Commutativity of $\oplus_k$).} \textit{
    For any $A, B \in \mathcal{P}_f(R)$,
    $A \oplus_k B = B \oplus_k A$.
}

We will prove that for any $A, B \in \mathcal{P}(R)$ and $n \in \N$,
$\eta_k^{(n)}(A, B) = \eta_k^{(n)}(B, A)$. The theorem
follows immediately from this statement.

We will use induction on $n$. Suppose $n = 1$. Then,
\[\begin{split}
    \eta_k(A, B)
    &= ((A \cup B) - (A \cap B), \{mk \mid m \in (A \cap B)\}) \\
    &= ((B \cup A) - (B \cap A), \{mk \mid m \in (B \cap A)\}) \\
    &= \eta_k(B, A)
\end{split}\]
This proves the base case. Now, suppose the hypothesis holds
for $n$. We will show that it holds for $n+1$.
\[\begin{split}
    \eta_k^{(n+1)}(A, B) &= \eta_k(\eta_k^{(n)}(A, B)) \\
    &= \eta_k(\eta_k^{(n)}(B, A)) \\
    &= \eta_k^{(n+1)}(B, A)
\end{split}\]
This proves the inductive step, completing the proof of the
lemma. $\blacksquare$

\textbf{Lemma 2.1.} \textit{
    Let $A, B \in \mathcal{P}_f(R)$. If $A \cap B = \emptyset$,
    then $A \oplus_k B = A \cup B$.
}

Let $A, B \in \mathcal{P}_f(R)$ with $A \cap B = \emptyset$.
Then,
\[\begin{split}
A \oplus_k B
&= ((A \cup B) - (A \cap B))
    \oplus_k \{mk \mid m \in (A \cap B)\} \\
&= ((A \cup B) - \emptyset)
    \oplus_k \{mk \mid m \in \emptyset\} \\
&= (A \cup B) \oplus_k \emptyset \\
\text{(by Lemma 1.1)}\quad
&= A \cup B
\end{split}\]
This proves the lemma. $\blacksquare$

\textbf{Lemma 2.2.} \textit{
    For any $A, B, C \in \mathcal{P}_f(R)$,
    if one or more of $A$, $B$, or $C$ is empty,
    then
    $(A \oplus_k B) \oplus_k C
    = A \oplus_k (B \oplus_k C)$.
}

Suppose $A = \emptyset$. Then,
\[\begin{split}
    (A \oplus_k B) \oplus_k C
    &= (\emptyset \oplus_k B) \oplus_k C \\
    &= B \oplus_k C \\
    &= \emptyset \oplus_k (B \oplus_k C) \\
    &= A \oplus_k (B \oplus_k C)
\end{split}\]
Suppose $B = \emptyset$. Then,
\[\begin{split}
    (A \oplus_k B) \oplus_k C
    &= (A \oplus_k \emptyset) \oplus_k C \\
    &= A \oplus_k C \\
    &= A \oplus_k (\emptyset \oplus_k C) \\
    &= A \oplus_k (B \oplus_k C)
\end{split}\]
Suppose $C = \emptyset$. Then,
\[\begin{split}
    (A \oplus_k B) \oplus_k C
    &= (A \oplus_k B) \oplus_k \emptyset \\
    &= A \oplus_k B \\
    &= A \oplus_k (B \oplus_k \emptyset) \\
    &= A \oplus_k (B \oplus_k C)
\end{split}\]
This proves the lemma. $\blacksquare$

\textbf{Lemma 2.3.} \textit{
    For any $A, B, C \in \mathcal{P}_f(R)$,
    if $\abs{A} = \abs{B} = \abs{C} = 1$,
    then
    $(A \oplus_k B) \oplus_k C
    = A \oplus_k (B \oplus_k C)$.
}

Let $A, B, C \in \mathcal{P}_f(R)$
with $\abs{A} = \abs{B} = \abs{C} = 1$.
Let $a, b, c \in R$ such that
$A = \{a\}$, $B = \{b\}$, and $C = \{c\}$.
We will fix $a$ and consider the various possible relationships
between $a$, $b$, and $c$.

Suppose $c = a$. Then,
\[\begin{split}
    (A \oplus_k B) \oplus_k C
    &= (\{a\} \oplus_k \{b\}) \oplus_k \{c\} \\
    &= (\{a\} \oplus_k \{b\}) \oplus_k \{a\} \\
    \text{(by Theorem 2.1)}\quad
    &= \{a\} \oplus_k (\{a\} \oplus_k \{b\}) \\
    \text{(by Theorem 2.1)}\quad
    &= \{a\} \oplus_k (\{b\} \oplus_k \{a\}) \\
    &= A \oplus_k (B \oplus_k C)
\end{split}\]
Therefore, if $c = a$, the lemma holds. We will now only
consider the cases where $c \neq a$.

Suppose $a = b$. Consider the following cases:
\begin{itemize}
    \item Suppose $c = ka$. Then,
    \[\begin{split}
        (A \oplus_k B) \oplus_k C
        &= (\{a\} \oplus_k \{b\}) \oplus_k \{c\} \\
        &= (\{a\} \oplus_k \{a\}) \oplus_k \{ka\} \\
        &= \{ka\} \oplus_k \{ka\} \\
        &= \{k^2a\} \\
        &= \{a\} \oplus_k \{a, ka\} \\
        &= \{a\} \oplus_k (\{a\} \oplus_k \{ka\}) \\
        &= A \oplus_k (B \oplus_k C)
    \end{split}\]
    \item Suppose $c \neq ka$. Then,
    \[\begin{split}
        (A \oplus_k B) \oplus_k C
        &= (\{a\} \oplus_k \{b\}) \oplus_k \{c\} \\
        &= (\{a\} \oplus_k \{a\}) \oplus_k \{c\} \\
        &= \{ka\} \oplus_k \{c\} \\
        &= \{ka, c\} \\
        &= \{a\} \oplus_k \{a, c\} \\
        &= \{a\} \oplus_k (\{a\} \oplus_k \{c\}) \\
        &= A \oplus_k (B \oplus_k C)
    \end{split}\]
\end{itemize}
Therefore, if $a = b$, then the lemma holds.
Using commutativity, we can rearrange $A \oplus_k (B \oplus_k C)$
to $(C \oplus_k B) \oplus_k A$ and apply the same reasoning
above to find that, if $b = c$, then
$(C \oplus_k B) \oplus_k A = C \oplus_k (B \oplus_k A)$,
or equivalently
$(A \oplus_k B) \oplus_k C = A \oplus_k (B \oplus_k C)$.

Suppose $a \neq b$, $b \neq c$, and $c \neq a$. Then,
\[\begin{split}
    (A \oplus_k B) \oplus_k C
    &= (\{a\} \oplus_k \{b\}) \oplus_k \{c\} \\
    &= \{a, b\} \oplus_k \{c\} \\
    &= \{a, b, c\}) \\
    &= \{a\} \oplus_k \{b, c\} \\
    &= \{a\} \oplus_k (\{b\} \oplus_k \{c\}) \\
    &= A \oplus_k (B \oplus_k C)
\end{split}\]
We have proved the lemma in the cases where:
\begin{enumerate}
    \item $a = b$
    \item $b = c$
    \item $c = a$
    \item $a \neq b$, $b \neq c$, and $c \neq a$
\end{enumerate}
If 1, 2, and 3 are false, then 4 is true. Therefore,
in all cases, the lemma holds.
$\blacksquare$

\textbf{Theorem 2.2 (Associativity of $\oplus_k$).} \textit{
    For any $A, B, C \in \mathcal{P}_f(R)$,
    $(A \oplus_k B) \oplus_k C = A \oplus_k (B \oplus_k C)$.
}

Let $n \in \N$ such that $n \geq \abs{A}$, $n \geq \abs{B}$, and
$n \geq \abs{C}$. We will refer to this property as $n$ being
an \textit{upper bound}, and we will use induction on $n$.

Let $A, B, C \in \mathcal{P}_f(R)$
such that 1 is an upper bound.
If any of $A$, $B$,
or $C$ is empty, then Lemma 2.2 completes the proof.
If they are all non-empty, then
$\abs{A} = \abs{B} = \abs{C} = 1$, so Lemma 2.3
completes the proof. Thus, the base case holds.

Suppose that the hypothesis holds for a value $n \in \N$.
We will show that it holds for $n + 1$.
Let $A, B, C \in \mathcal{P}_f(R)$ such that $n + 1$
is an upper bound.
If any of $A$, $B$, or $C$ are empty, then applying
Lemma 2.3 completes the proof, so we will assume they are
all non-empty. Let $a \in A$, $b \in B$, and $c \in C$.
Let $A' = A - \{a\}$, $B' = B - \{b\}$, and $C' = C - \{c\}$.
$n \geq \abs{A'}$, $n \geq \abs{B'}$, $n \geq \abs{C'}$,
and $n \geq 1 = \abs{\{a\}} = \abs{\{b\}} = \abs{\{c\}}$,
so associativity holds for all of these sets. Therefore,
\[\begin{split}
    (A \oplus_k B) \oplus_k C
    &= ((A' \cup \{a\}) \oplus_k (B' \cup \{b\}))
        \oplus_k (C' \cup \{c\}) \\
    \text{(by Lemma 2.1)}\quad
    &= ((A' \oplus_k \{a\}) \oplus_k (B' \oplus_k \{b\}))
        \oplus_k (C' \oplus_k \{c\}) \\
    \text{(by inductive hypothesis)}\quad
    &= (((A' \oplus_k \{a\}) \oplus_k B') \oplus_k \{b\})
        \oplus_k (C' \oplus_k \{c\}) \\
    \text{(by inductive hypothesis)}\quad
    &= ((A' \oplus_k \{a\}) \oplus_k B') \oplus_k
        (\{b\} \oplus_k (C' \oplus_k \{c\})) \\
    \text{(by inductive hypothesis)}\quad
    &= (A' \oplus_k \{a\}) \oplus_k (B' \oplus_k
        (\{b\} \oplus_k (C' \oplus_k \{c\}))) \\
    \text{(by inductive hypothesis)}\quad
    &= (A' \oplus_k \{a\}) \oplus_k ((B' \oplus_k
        \{b\}) \oplus_k (C' \oplus_k \{c\})) \\
    &= A \oplus_k (B \oplus_k C) \\
\end{split}\]
The chunk above here is not valid, need to fix.

This proves the inductive step, completing the proof of
the theorem.
$\blacksquare$

\end{document}
