\documentclass{article}
\usepackage[utf8]{inputenc}
\usepackage{amsmath,amssymb}
\usepackage{enumitem}
\usepackage{graphicx,tikz}
\usepackage{geometry}

\geometry{margin=1in}
\pagestyle{empty}
\setlength{\parindent}{0pt}
\setlength{\parskip}{0.5em}

\newcommand{\code}[1]{{\fontfamily{pcr}\selectfont #1}}
\newcommand{\abs}[1]{\left|#1\right|}
\newcommand{\srs}[1]{\{#1\}_1^\infty}
\newcommand{\floor}[1]{\left\lfloor#1\right\rfloor}
\newcommand{\ceil}[1]{\left\lceil#1\right\rceil}
\newcommand{\sumt}{\text{sum}}

\newcommand{\env}[2]{\begin{#1}#2\end{#1}}
\newcommand{\spl}[1]{\begin{split}#1\end{split}}
\newcommand{\mat}[1]{\begin{bmatrix}#1\end{bmatrix}}

\newcommand{\zee}{\mathbb{Z}}
\newcommand{\Q}{\mathbb{Q}}
\newcommand{\arr}{\mathbb{R}}
\newcommand{\C}{\mathbb{C}}
\newcommand{\N}{\mathbb{N}}

\begin{document}

Let $R$ be a ring, and let $k \in R$.

\section{Introduction of $\eta_k$ and $\oplus_k$}

\textbf{Definition 1.1.} For $A, B \in \mathcal{P}(R)$, we define
$\eta_k: \mathcal{P}(R) \times \mathcal{P}(R)
\rightarrow \mathcal{P}(R) \times \mathcal{P}(R)$ as
\[\eta_k(A, B) = ((A \cup B) - (A \cap B), \{nk \mid n \in (A \cap B)\})\]
For $0 \leq n \in \N$, $\eta_k^{(n)}$ denotes $\eta_k$ composed
with itself $n$ times. $\pi_1 \circ \eta_k^{(n)}$ denotes the first
element of the output of $\eta_k^{(n)}$, and $\pi_2 \circ \eta_k^{(n)}$
denotes the second.

\textbf{Lemma 1.1.} \textit{
    For any $A \in \mathcal{P}(R)$ and for any $n \in \zee$
    with $n \geq 0$,
    $\eta_k^{(n)}(A, \emptyset) = (A, \emptyset)$.
}

Suppose $n = 0$. Trivially,
$\eta_k^{(0)}(A, \emptyset) = (A, \emptyset)$.

For $n > 0$, we will use induction on $n$.
Suppose $n = 1$. Then,
\[\begin{split}
    \eta_k(A, \emptyset)
    &= ((A \cup \emptyset) - (A \cap \emptyset),
    \{nk \mid n \in (A \cap \emptyset)\}) \\
    &= (A - \emptyset, \{nk \mid n \in \emptyset\}) \\
    &= (A, \emptyset)
\end{split}\]
This proves the base case.

Now, suppose the hypothesis holds
for $n$. We will show that it holds for $n+1$.
\[\begin{split}
    \eta_k^{(n+1)}(A, \emptyset)
    &= \eta_k(\eta_k^{(n)}(A, \emptyset)) \\
    \text{(by inductive hypothesis)}\quad
    &= \eta_k(A, \emptyset) \\
    \text{(by $n=1$ case)}\quad
    &= (A, \emptyset)
\end{split}\]
This proves the inductive step, completing the proof of the
lemma. $\blacksquare$

\textbf{Lemma 1.2.} \textit{
    For any $A, B \in \mathcal{P}(R)$, if there exists a
    nonnegative integer $N$ such that
    $\pi_2 \circ \eta_k^{(N)} = \emptyset$,
    then $\exists C \in \mathcal{P}(R)$ such that,
    for any integer $n$ with $n \geq N$,
    $\eta_k^{(n)} = (C, \emptyset)$.
}

Let $C = \pi_1 \circ \eta_k^{(N)}(A, B)$. Then,
\[\begin{split}
    \eta_k^{(n)}(A, B)
    &= \eta_k^{(n-N)} \circ \eta_k^{(N)}(A, B) \\
    &= \eta_k^{(n-N)}\big(
        \pi_1 \circ \eta_k^{(N)}(A, B), \emptyset
    \big) \\
    &= \eta_k^{(n-N)}(C, \emptyset) \\
    \text{(by Lemma 1.1)}\quad
    &= (C, \emptyset)
\end{split}\]
This proves the lemma. $\blacksquare$

\textbf{Definition 1.2.} $\mathcal{P}_f(R)$ denotes the
set of all elements of $\mathcal{P}(R)$ which contain
a finite number of elements.

\textbf{Theorem 1.1 (Convergence of $\eta_k^{(n)}$).} \textit{
    For any $A, B \in \mathcal{P}_f(R)$, there exists
    some nonnegative integer $N$
    such that, for some $C \in \mathcal{P}_f(R)$,
    $N \leq n \in \zee \implies \eta_k^{(n)}(A, B) = (C, \emptyset)$.
}

We will use contradiction to show that there exists a nonnegative
integer $N$ for which $\pi_2 \circ \eta_k^{(N)} = \emptyset$.
By Lemma 1.2, this statement is equivalent to Theorem 1.1.

Suppose $A, B \in \mathcal{P}(R)$ for which no such $N$ exists.
Let $A', B' \in \mathcal{P}(R)$ with $\eta_k(A, B) = (A', B')$.
If $A \cap B = \emptyset$, then $\pi_2 \circ \eta_k(A, B)
= B' = \emptyset$.
This is a contradiction, so we will assume $A \cap B$ is not empty.

We will now consider the size of $A'$
\[\begin{split}
    \abs{A'} &= \abs{A \cup B} - \abs{A \cap B}
    = (\abs{A} + \abs{B} - \abs{A \cap B}) - \abs{A \cap B} \\
    &= \abs{A} + \abs{B} - 2\abs{A \cap B}
\end{split}\]
And of $B'$
\[\abs{B'} = \abs{A \cap B}\]
Adding these together, we find
\[\begin{split}
\abs{A'} + \abs{B'}
&= \abs{A} + \abs{B} - 2\abs{A \cap B} + \abs{A \cap B}
= \abs{A} + \abs{B} - \abs{A \cap B} \\
&< \abs{A} + \abs{B}
\end{split}\]
If there exists a nonnegative integer $N$ for which
$\pi_2 \circ \eta_k^{(N)}(A', B') = \emptyset$, then
$\pi_2 \circ \eta_k^{(N+1)}(A, B) = \emptyset$. This is
a contradiction, so no such $N$ exists for $A'$ and $B'$.
Therefore, we can repeatedly apply the same reasoning, replacing
$A$ and $B$ with $A'$ and $B'$. Doing so, we find
\[\abs{A} + \abs{B}
> \abs{\pi_1 \circ \eta_k(A, B)} + \abs{\pi_1 \circ \eta_k(A, B)}
> \abs{\pi_1 \circ \eta_k^{(2)}(A, B)}
    + \abs{\pi_1 \circ \eta_k^{(2)}(A, B)}
> \ldots\]
For all $n \in \zee$ with $n \geq 0$, as $n$ increases,
$f(n) = \abs{\pi_1 \circ \eta_k^{(n)}(A, B)}
+ \abs{\pi_1 \circ \eta_k^{(n)}(A, B)}$ strictly decreases.
However, it is an integer and is bounded above by
$\abs{A} + \abs{B}$, so for some value of $n$, $f(n) < 0$.
Since the size of a set can never be negative, this is a contradiction.
Therefore, there must exist some nonnegative integer $N$ for which
$\pi_2 \circ \eta_k^{(N)} = \emptyset$. This proves the theorem.
$\blacksquare$

\textbf{Corollary 1.1.} \textit{
    For any $A, B \in \mathcal{P}_f(R)$, let $N$ be the
    least nonnegative integer such that
    $\pi_2 \circ \eta_k^{(N)} = \emptyset$. Then,
    $\exists C \in \mathcal{P}_f(R)$ such that
    $N \leq n \in \zee \implies \eta_k^{(n)}(A, B) = (C, \emptyset)$.
}

By Theorem 1.1, such a value exists of $N$ exists.
Since $N$ is bounded below by 0, we can find a minimum value for which
$\pi_2 \circ \eta_k^{(N)} = \emptyset$. Applying Lemma 1.2 to the
minimum value proves the corollary. $\blacksquare$

\textbf{Definition 1.3.} For $A, B \in \mathcal{P}_f(R)$,
let $N$ be the least value for which
$\pi_2 \circ \eta_k^{(N)}(A, B) = \emptyset$. Then, we define
the binary operator $\oplus_k: \mathcal{P}_f(R) \times \mathcal{P}_f(R)
\rightarrow \mathcal{P}_f(R)$ as
\[A \oplus_k B = \pi_1 \circ \eta_k^{(N)}(A, B)\]
This is equivalent to the following recursive definition:
\[A \oplus_k B = \begin{cases}
    A & B = \emptyset \\
    ((A \cup B) - (A \cap B)) \oplus_k \{nk \mid n \in (A \cap B)\}
        & A \neq \emptyset, B \neq \emptyset
\end{cases}\]

\section{Properties of $\oplus_k$}

\textbf{Theorem 2.1 (Commutativity of $\oplus_k$).} \textit{
    For any $A, B \in \mathcal{P}_f(R)$,
    $A \oplus_k B = B \oplus_k A$.
}

We will prove that for any $A, B \in \mathcal{P}(R)$ and $n \in \N$,
$\eta_k^{(n)}(A, B) = \eta_k^{(n)}(B, A)$. The theorem
follows immediately from this statement.

We will use induction on $n$. Suppose $n = 1$. Then,
\[\begin{split}
    \eta_k(A, B)
    &= ((A \cup B) - (A \cap B), \{nk \mid n \in (A \cap B)\}) \\
    &= ((B \cup A) - (B \cap A), \{nk \mid n \in (B \cap A)\}) \\
    &= \eta_k(B, A)
\end{split}\]
This proves the base case. Now, suppose the hypothesis holds
for $n$. We will show that it holds for $n+1$.
\[\begin{split}
    \eta_k^{(n+1)}(A, B) &= \eta_k(\eta_k^{(n)}(A, B)) \\
    &= \eta_k(\eta_k^{(n)}(B, A)) \\
    &= \eta_k^{(n+1)}(B, A)
\end{split}\]
This proves the inductive step, completing the proof of the
lemma. $\blacksquare$

\textbf{Theorem 2.2 (Associativity of $\oplus_k$).} \textit{
    For any $A, B, C \in \mathcal{P}_f(R)$,
    $(A \oplus_k B) \oplus_k C = A \oplus_k (B \oplus_k C)$.
}

Just take my word for it. $\blacksquare$

\end{document}
