\documentclass{article}
\usepackage[utf8]{inputenc}
\usepackage{amsmath,amssymb}
\usepackage{enumitem}
\usepackage{graphicx,tikz}
\usepackage{geometry}

\geometry{margin=1in}
\pagestyle{empty}
\setlength{\parindent}{0pt}
\setlength{\parskip}{0.5em}

\newcommand{\code}[1]{{\fontfamily{pcr}\selectfont #1}}
\newcommand{\abs}[1]{\left|#1\right|}
\newcommand{\srs}[1]{\{#1\}_1^\infty}
\newcommand{\floor}[1]{\left\lfloor#1\right\rfloor}
\newcommand{\ceil}[1]{\left\lceil#1\right\rceil}

\newcommand{\env}[2]{\begin{#1}#2\end{#1}}
\newcommand{\spl}[1]{\begin{split}#1\end{split}}
\newcommand{\mat}[1]{\begin{bmatrix}#1\end{bmatrix}}

\newcommand{\zee}{\mathbb{Z}}
\newcommand{\Q}{\mathbb{Q}}
\newcommand{\arr}{\mathbb{R}}
\newcommand{\C}{\mathbb{C}}
\newcommand{\N}{\mathbb{N}}

\begin{document}

Let $R$ be a ring.

\textbf{Definition 1.1.} For $A, B \in \mathcal{P}(R)$ and $k \in R$,
\[A \oplus_k B = \begin{cases}
    A & B = \emptyset \\
    B & A = \emptyset \\
    ((A \cup B) - (A \cap B)) \oplus_k \{nk \mid n \in (A \cap B)\}
        & A \neq \emptyset, B \neq \emptyset
\end{cases}\]

Note that if $A = \emptyset$, then
\[\begin{split}
    ((A \cup B) - (A \cap B)) \oplus_k \{nk \mid n \in (A \cap B)\}
    &= ((\emptyset \cup B) - (\emptyset \cap B))
        \oplus_k \{nk \mid n \in (\emptyset \cap B)\} \\
    &= (B - \emptyset) \oplus_k \{nk \mid n \in \emptyset\} \\
    &= B \oplus_k \emptyset = B = \emptyset \oplus_k B \\
    &= A \oplus_k B
\end{split}\]
And if $B = \emptyset$, then
\[\begin{split}
    ((A \cup B) - (A \cap B)) \oplus_k \{nk \mid n \in (A \cap B)\}
    &= ((A \cup \emptyset) - (A \cap \emptyset))
        \oplus_k \{nk \mid n \in (A \cap \emptyset)\} \\
    &= (A - \emptyset) \oplus_k \{nk \mid n \in \emptyset\} \\
    &= A \oplus_k \emptyset \\
    &= A \oplus_k B
\end{split}\]
Therefore, in all cases,
$A \oplus_k B
= ((A \cup B) - (A \cap B)) \oplus_k \{nk \mid n \in (A \cap B)\}$.

Because it is recursive, it is not immediately clear that it
produces a result for all inputs. We will show that, as long
as its inputs are finite sets, it will produce an output
in a finite number of steps.

\textbf{Definition 1.2.} For $A, B \in \mathcal{P}(R)$ and $k \in R$,
$A \oplus_k B$ is \textit{defined} if it can be computed in a finite
number of steps.

\textbf{Theorem 1.1.} \textit{
    For all finite $A, B \in \mathcal{P}(R)$
    and for any $k \in R$,
    $A \oplus_k B$ is defined.
}

Suppose that for $A, B \in \mathcal{P}(R)$, $A \oplus_k B$ is not
defined. Clearly, $A \neq \emptyset \neq B$. If $A \cap B = \emptyset$,
then $A \oplus_k B = (A \cup B) \oplus_k \emptyset = A \cup B$, which
contradicts the assumption that $A \oplus_k B$ is not defined.
Therefore, $A \cap B \neq \emptyset$.

Let $A' = ((A \cup B) - (A \cap B))$ and
$B' = \{nk \mid n \in (A \cap B)\}$. $A \oplus_k B = A' \oplus_k B'$,
so $A' \oplus_k B'$ also must not be defined. Now, consider the size
of $A'$
\[\begin{split}
\abs{A'} &= \abs{A \cup B} - \abs{A \cap B}
= (\abs{A} + \abs{B} - \abs{A \cap B}) - \abs{A \cap B} \\
&= \abs{A} + \abs{B} - 2\abs{A \cap B}
\end{split}\]
And of $B'$
\[\abs{B'} = \abs{A \cap B}\]
Adding these together, we find
\[\begin{split}
\abs{A'} + \abs{B'}
&= \abs{A} + \abs{B} - 2\abs{A \cap B} + \abs{A \cap B}
= \abs{A} + \abs{B} - \abs{A \cap B} \\
&< \abs{A} + \abs{B}
\end{split}\]
Since $A' \oplus_k B'$ is not defined, we can apply the same reasoning
repeatedly, replacing $A$ and $B$ with $A'$ and $B'$ respectively.
Doing this, we find an infinite sequence of pairs of sets, each of
which have a smaller sum of sizes than the previous pair.
However, the least possible sum of sizes for a pair of sets is zero,
so it cannot infinitely decrease. This is a contradiction---therefore,
$A \oplus_k B$ must be defined. $\blacksquare$

\textbf{Corollary 1.1.} \textit{
    For all finite $A, B \in \mathcal{P}(R)$
    and for any $k \in R$,
    $A \oplus_k B$ is finite.
}

Let $A, B \in \mathcal{P}(R)$ be finite, and let $k \in R$.
Then, with $A' = ((A \cup B) - (A \cap B))$ and
$B' = \{nk \mid n \in (A \cap B)\}$, we find

\[\begin{split}
    \abs{A'} + \abs{B'}
    &= \abs{A} + \abs{B} - 2\abs{A \cap B} + \abs{A \cap B}
    = \abs{A} + \abs{B} - \abs{A \cap B} \\
    &\leq \abs{A} + \abs{B}
\end{split}\]

Repeatedly applying this
reasoning with $A'$ and $B'$ replacing $A$ and $B$ respectively,
we find that this holds after any number of iterations. Since
$A \oplus_k B$ is defined, there exists a set
$C \in \mathcal{P}(R)$ with
$A \oplus_k B = C \oplus_k \emptyset = C$. For such a $C$,
$\abs{C} = \abs{C} + \abs{\emptyset} \leq \abs{A} + \abs{B}$,
so $A \oplus_k B = C$ is finite. $\blacksquare$

Next, we will show that $\oplus_k$ is commutative.

\textbf{Theorem 1.2.} \textit{
    For all finite $A, B \in \mathcal{P}(R)$
    and for any $k \in R$,
    $A \oplus_k B = B \oplus_k A$.
}

Let $A, B \in \mathcal{P}(R)$, $k \in R$. Then,
\[\begin{split}
    A \oplus_k B
    &= ((A \cup B) - (A \cap B)) \oplus_k \{nk \mid n \in (A \cap B)\} \\
    &= ((B \cup A) - (B \cap A)) \oplus_k \{nk \mid n \in (B \cap A)\} \\
    &= B \oplus_k A
\end{split}\]
This proves the theorem. $\blacksquare$

\textbf{Theorem 1.3.} \textit{
    For all finite $A, B, C \in \mathcal{P}(R)$
    and for any $k \in R$,
    $(A \oplus_k B) \oplus C = A \oplus_k (B \oplus C)$.
}

\end{document}
