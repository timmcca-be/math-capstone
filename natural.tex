\documentclass{article}
\usepackage[utf8]{inputenc}
\usepackage{amsmath,amssymb}
\usepackage{enumitem}
\usepackage{graphicx,tikz}
\usepackage{geometry}

\geometry{margin=1in}
\pagestyle{empty}
\setlength{\parindent}{0pt}
\setlength{\parskip}{0.5em}

\newcommand{\code}[1]{{\fontfamily{pcr}\selectfont #1}}
\newcommand{\abs}[1]{\left|#1\right|}
\newcommand{\srs}[1]{\{#1\}_1^\infty}
\newcommand{\floor}[1]{\left\lfloor#1\right\rfloor}
\newcommand{\ceil}[1]{\left\lceil#1\right\rceil}

\newcommand{\env}[2]{\begin{#1}#2\end{#1}}
\newcommand{\spl}[1]{\begin{split}#1\end{split}}
\newcommand{\mat}[1]{\begin{bmatrix}#1\end{bmatrix}}

\newcommand{\zee}{\mathbb{Z}}
\newcommand{\Q}{\mathbb{Q}}
\newcommand{\arr}{\mathbb{R}}
\newcommand{\C}{\mathbb{C}}
\newcommand{\N}{\mathbb{N}}

\begin{document}

\setcounter{section}{4}
\section{$\oplus_k$ over $\N$}

\textbf{Definition 5.1.} Define $R \subseteq \N$ by
\[R = \{n \in \N \mid k \nmid n\}\]
Let $R = \{r_1, r_2, r_3, \ldots\}$ with $r_1 < r_2 < r_3 < \ldots$.

\textbf{Definition 5.2.} Let $n \in \N$ and let $A \subseteq \N$.
For all $i \in \zee$
with $i \geq 0$, let $z_i = 0$ if $nk^i \not\in A$,
and let $z_i = 1$ if $nk^i \in A$.
Define $\beta': \N \times \mathcal{P}(\N) \rightarrow \N$
by
\[\beta'(n, A) = \sum_{i = 0}^\infty z_i \cdot 2^{i}\]
This is a more general form of $\beta$ from Definition 4.2.
For $A \in \zee_p'$, $\beta(A)$ is generally
the same as $\beta'(1, A)$.

\textbf{Definition 5.3.} Let
$p_1, p_2, p_3, \ldots \in \N$ denote the primes in increasing
order---that is, $p_1 = 2$, $p_2 = 3$, $p_3 = 5$, etc.
Let $A \subseteq \N$.
Then, define $\alpha: \mathcal{P}(\N) \rightarrow \N$ by
\[\alpha(A) = \prod_{i=1}^\infty p_i^{\beta'(r_i, A)}\]
We will consider an example. Suppose $k = 4$, and let
$A = \{1, 3, 4, 9, 16, 48\}$.
\begin{itemize}
    \item Consider $i = 1$. We are interested in $r_1 = 1$.
    $r_1k^0 = 1 \in A$,
    $r_1k^1 = 4 \in A$, and $r_1k^2 = 16 \in A$.
    Therefore,
    \[\beta'(r_1, A) = \beta'(1, A) = \text{0b111} = 7\]
    \item Consider $i = 2$. We are interested in $r_2 = 2$.
    Since there is no $n \in \N$ for which $2k^n \in A$,
    we find
    \[\beta'(r_2, A) = \beta'(2, A) = 0\]
    \item Consider $i = 3$. We are interested in $r_3 = 3$.
    $r_3k^0 = 3 \in A$ and $r_3k^2 = 48 \in A$.
    Therefore,
    \[\beta'(r_3, A) = \beta'(3, A) = \text{0b101} = 5\]
    \item Consider $i = 7$. We are interested in $r_7 = 9$.
    $r_7k^0 = 9 \in A$. Therefore,
    \[\beta'(r_7, A) = \beta'(9, A) = \text{0b1} = 1\]
\end{itemize}
Therefore,
\[\begin{split}
    \alpha(A) &= p_1^7 \cdot p_3^5 \cdot p_7^1 \\
    &= 2^7 \cdot 5^5 \cdot 17 \\
    &= 6800000
\end{split}\]

\textbf{Theorem 5.1.} \textit{Let $A, B \subseteq \N$.
Then, $\alpha(A \oplus_k B) = \alpha(A) \cdot \alpha(B)$.}

Very similar to Theorem 4.1, I'll write it out later.
$\blacksquare$

\end{document}
