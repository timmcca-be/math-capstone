\documentclass{article}
\usepackage[utf8]{inputenc}
\usepackage{amsmath,amssymb}
\usepackage{graphicx,tikz}
\usepackage{geometry}

\geometry{margin=1in}
\setlength{\parindent}{0pt}
\pagestyle{empty}

\newcommand{\code}[1]{{\fontfamily{pcr}\selectfont #1}}
\newcommand{\abs}[1]{\left|#1\right|}
\newcommand{\srs}[1]{\{#1\}_1^\infty}

\newcommand{\env}[2]{\begin{#1}#2\end{#1}}
\newcommand{\spl}[1]{\begin{split}#1\end{split}}
\newcommand{\mat}[1]{\begin{bmatrix}#1\end{bmatrix}}

\newcommand{\zee}{\mathbb{Z}}
\newcommand{\Q}{\mathbb{Q}}
\newcommand{\arr}{\mathbb{R}}
\newcommand{\C}{\mathbb{C}}
\newcommand{\N}{\mathbb{N}}

\newcommand{\sumt}{\text{sum}}

\begin{document}

\section{Introduction}

For $A, B \in \mathcal{P}(\zee)$ and $k \in \zee$, consider
\[A \oplus_k B = \begin{cases}
    A & B = \emptyset \\
    B & A = \emptyset \\
    ((A \cup B) - (A \cap B)) \oplus_k \{kn \mid n \in (A \cap B)\}
        & A \neq \emptyset, B \neq \emptyset
\end{cases}\]

For $A \in \mathcal{P}(\zee)$, let $\sumt(A)$ be the sum of all distinct
elements of $A$. Let $\sumt(\emptyset) = 0$.

A few remarks:
\begin{itemize}
    \item The problem is uninteresting for $k = 0$ and $k = 1$, as
    \begin{itemize}
        \item $A \oplus_0 B = ((A \cup B) - (A \cap B)) \cup \{0\}$.
        \item $A \oplus_1 B = A \cup B$.
    \end{itemize}
    \item $\sumt(A) + \sumt(B) = \sumt(A \oplus_2 B)$.
    \item Since $\cup$ and $\cap$ are commutative, $\oplus_k$ is
    commutative for any $k \in \zee$.
\end{itemize}

\textbf{Conjecture.} \textit{For all $k \in \zee$.
\begin{enumerate}
    \item $A \oplus_k B$ is defined for all $A, B \in \mathcal{P}(\zee)$.
    \item $\oplus_k$ is associative.
\end{enumerate}}

If this conjecture is proven true, there are two directions that it can be
taken, detailed in the following sections.

\section{Zero sum sets with $\oplus_2$}

Let $Z = \{A \mid A \in \mathcal{P}(\zee), \sumt(A) = 0\}$. $Z$ is closed
on $\oplus_2$, and $\emptyset$ is the identity. It may be interesting to
investigate the factorization of elements of $Z$ with respect to
$\oplus_2$ and to find the properties of irreducible elements of $Z$.
This may also provide some insight regarding the subset sum problem.

\section{Diffie--Hellman}

For $A \in \mathcal{P}(\zee)$, $k \in \zee$, and $n$ a non-negative integer,
define $A_k^n$ as
\[\underbrace{A \oplus_k A \oplus_k \ldots \oplus_k A}_{n\text{ times}}\]
Consider the problem of computing $n$ given $A$, $k$, and $A_k^n$.
We aim to find a value of $k$ such that there exist infinite sets
$A$ for which the problem is intractable for sufficiently large $n$.
If $k = 0$ or $1$, then $A_k^n$ does not change as $n$ increases
past a certain value, so we will not consider these cases. If $k = -1$,
then $A_{-1}^n$ will cycle as $n$ increases, so we will also discard this
case. If $k = 2$ and $\sumt(A) \neq 0$, then $n = \sumt(A_2^n) / \sumt(A)$,
so finding $n$ is trivial. The tractability in other cases is not
immediately clear.
\\[0.5em]
Suppose such a $k$ exists. If the conjecture is true, then the following
adaptation of the Diffie--Hellman key exchange can be used to securely
exchange cryptographic keys:
\begin{enumerate}
    \item Alice and Bob publicly agree to use a set
    $Q \in \mathcal{P}(\zee)$.
    \item Alice chooses a secret non-negative integer $a$, then sends Bob
    $A = Q_k^a$.
    \item Bob chooses a secret non-negative integer $b$, then sends Alice
    $B = Q_k^b$.
    \item Alice computes $S = B_k^a$.
    \item Bob computes $S = A_k^b$.
    \item Alice and Bob now share a secret (the set $S = Q_k^{ab}$).
\end{enumerate}

\end{document}
