\documentclass{article}
\usepackage[utf8]{inputenc}
\usepackage{amsmath,amssymb}
\usepackage{enumitem}
\usepackage{graphicx,tikz}
\usepackage{geometry}

\geometry{margin=1in}
\pagestyle{empty}
\setlength{\parindent}{0pt}
\setlength{\parskip}{0.5em}

\newcommand{\code}[1]{{\fontfamily{pcr}\selectfont #1}}
\newcommand{\abs}[1]{\left|#1\right|}
\newcommand{\srs}[1]{\{#1\}_1^\infty}
\newcommand{\floor}[1]{\left\lfloor#1\right\rfloor}
\newcommand{\ceil}[1]{\left\lceil#1\right\rceil}

\newcommand{\env}[2]{\begin{#1}#2\end{#1}}
\newcommand{\spl}[1]{\begin{split}#1\end{split}}
\newcommand{\mat}[1]{\begin{bmatrix}#1\end{bmatrix}}
\newcommand{\tim}{\;\text{tim}\;}

\newcommand{\zee}{\mathbb{Z}}
\newcommand{\Q}{\mathbb{Q}}
\newcommand{\arr}{\mathbb{R}}
\newcommand{\C}{\mathbb{C}}
\newcommand{\N}{\mathbb{N}}

\begin{document}

\setcounter{section}{4}
\section{Timular Arithmetic}

Let $x \in \N$.

\textbf{Definition 5.1.} Let $a, b \in \zee$.
Then, $a$ is \textit{similar to $b$ timulo} $x$,
denoted $a \sim b \tim x$,
if there exist $c_1, c_2, \ldots, c_r \in \zee$
with $c_1 \geq 0, c_2 \geq 0, \ldots, c_r \geq 0$
such that, for some $m_1, m_2, \ldots, m_r \in \zee$
\[a = c_1x^{m_1} + c_2x^{m_2} + \ldots + c_rx^{m_r}\]
and for some $n_1, n_2, \ldots, n_r \in \zee$,
\[b = c_1x^{n_1} + c_2x^{n_2} + \ldots + c_rx^{n_r}\]


\textbf{Lemma 5.1.} \textit{Similarity timulo $x$ is a
tolerance relation.}

First, we will show that it is reflexive. Let $a \in \zee$.
Then, $a = ax^0 \equiv ax^0 \equiv a \tim x$.

Symmetry is obvious, as similarity is defined without
regard to the order of $a$ and $b$. This proves the lemma.
$\blacksquare$


\textbf{Definition 5.2.} Let $a, b \in \zee$.
Then, $a$ is \textit{congruent to $b$ timulo} $x$, denoted
$a \equiv b \tim x$, if there exist some $d_1, d_2, \ldots, d_s \in \zee$
such that
\[a \sim d_1 \sim d_2 \sim \ldots \sim d_s \sim b \tim x\]

\textbf{Theorem 5.1.} \textit{Congruence timulo $x$ is an
equivalence relation.}

First, we will show that it is reflexive. Let $a \in \zee$.
By Lemma 5.1, $a \sim a \tim x$, so $a \equiv a \tim x$.

Next, we will show that it is symmetric. Let $a, b \in \zee$.
Let $d_1, d_2, \ldots, d_s \in \zee$ such that
\[a \sim d_1 \sim d_2 \sim \ldots \sim d_s \sim b \tim x\]
Then, by Lemma 5.1,
\[b \sim d_s \sim d_{s-1} \sim \ldots \sim d_1 \sim a \tim x\]
Therefore, $b \equiv a \tim x$.

Transitivity clearly follows from Definition 5.2. This proves the
theorem. $\blacksquare$

\textbf{Theorem 5.2.} \textit{Let $a \in \zee$.
Then, $\exists b \in \zee$ with $0 \leq b < x$
such that $a \equiv b \tim x$.}

Obvious descent argument. $\blacksquare$

\textbf{Definition 5.3.} Let $a \in \zee$.
$b \in \zee$ is the \textit{canonical form of $a$ timulo $x$}
if $0 \leq b < x$.

\textbf{Definition 5.4.} Let $a \in \zee$. The
\textit{congruence class of $a$ timulo $x$},
denoted $(a:x)$, is the set of all $b \in \zee$
such that $a \equiv b \tim x$.

\textbf{Lemma 5.2.} \textit{Let $a, a' \in \zee$ be
similar timulo $x$, and let $b \in \zee$. Then,
$(a + b) \sim (a' + b) \tim x$.}

Let $a = c_1x^{m_1} + c_2x^{m_2} + \ldots + c_rx^{m_r}$,
and let
$a' = c_1x^{n_1} + c_2x^{n_2} + \ldots + c_rx^{n_r}$.
Then,
\[\begin{split}
    a + b &= (c_1x^{m_1} + c_2x^{m_2} + \ldots + c_rx^{m_r}) + b \\
    &= c_1x^{m_1} + c_2x^{m_2} + \ldots + c_rx^{m_r} + bx^0 \\
    &\sim c_1x^{n_1} + c_2x^{n_2} + \ldots + c_rx^{n_r} + bx^0 \\
    &\sim (c_1x^{n_1} + c_2x^{n_2} + \ldots + c_rx^{n_r}) + b \\
    &\sim a' + b \tim x
\end{split}\]
This proves the lemma. $\blacksquare$

\newpage
\textbf{Lemma 5.3.} \textit{Let $a, a' \in \zee$ be
congruent timulo $x$, and let $b \in \zee$. Then,
$a + b \equiv a' + b \tim x$.}

Let $d_1, d_2, \ldots, d_s \in \zee$
such that $a \sim d_1 \sim d_2 \sim \ldots \sim d_s \sim a' \tim x$.
Then, by Lemma 5.2,
\[(a + b) \sim (d_1 + b) \sim (d_2 + b) \sim \ldots
\sim (d_s + b) \sim (a' + b) \tim x\]
Therefore, $a + b \equiv a' + b \tim x$.
$\blacksquare$

\textbf{Lemma 5.4.} \textit{Let $a, a' \in \zee$ be
similar timulo $x$, and let $b \in \zee$. Then,
$ab \sim a'b \tim x$.}

Let $a = c_1x^{m_1} + c_2x^{m_2} + \ldots + c_rx^{m_r}$,
and let
$a' = c_1x^{n_1} + c_2x^{n_2} + \ldots + c_rx^{n_r}$.
Then,
\[\begin{split}
    ab &= (c_1x^{m_1} + c_2x^{m_2} + \ldots + c_rx^{m_r})b \\
    &= bc_1x^{m_1} + bc_2x^{m_2} + \ldots + bc_rx^{m_r} \\
    &\sim bc_1x^{n_1} + bc_2x^{n_2} + \ldots + bc_rx^{n_r} \\
    &\sim (c_1x^{n_1} + c_2x^{n_2} + \ldots + c_rx^{n_r})b \\
    &\sim a'b \tim x
\end{split}\]
This proves the lemma. $\blacksquare$

\textbf{Lemma 5.5.} \textit{Let $a, a' \in \zee$ be
congruent timulo $x$, and let $b \in \zee$. Then,
$ab \equiv a'b \tim x$.}

Let $d_1, d_2, \ldots, d_s \in \zee$
such that $a \sim d_1 \sim d_2 \sim \ldots \sim d_s \sim a' \tim x$.
Then, by Lemma 5.2,
\[ab \sim d_1b \sim d_2b \sim \ldots
\sim d_sb \sim a'b \tim x\]
Therefore, $ab \equiv a'b \tim x$.
$\blacksquare$

\textbf{Definition 5.5.} $T(x)$ denotes the set of all
congruence classes timulo $x$.

\textbf{Definition 5.6.} Let $a, b \in \zee$. Define
$(a:x) + (b:x)$ by $(a+b:x)$, and define $(a:x) \cdot (b:x)$
by $(ab:x)$.

\textbf{Definition 5.7.} Let $A \in \zee_p'$. For all $i \in \zee$
with $0 \leq i \leq p - 2$, let $z_i = 0$ if $k^i \not\in A$,
and let $z_i = 1$ if $k^i \in A$. Then, define
$\beta: \mathcal{P}(\zee_p') \rightarrow T(2^{p-1})$ by
\[\beta(A) = \sum_{i = 0}^{p-2} z_i \cdot 2^{i}\]
In other words, consider the binary representation of $\beta(A)$.
The least significant bit is 1 if $k^0 \in A$, and 0 otherwise.
The next bit is 1 if $k^1 \in A$ and 0 otherwise.
This continues to the most significant bit,
which is 1 if $k^{p-2} \in A$ and 0 otherwise.

\textbf{Theorem 5.3.} \textit{Let $A, B \in \zee_p'$.
Then, $\beta(A \oplus_k B) = \beta(A) + \beta(B)$.}

\end{document}
